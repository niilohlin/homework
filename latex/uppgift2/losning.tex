\documentclass[a4paper]{article}
\usepackage{amsmath}
\usepackage[swedish]{babel}
\usepackage[utf8]{inputenc}

\begin{document}
\title{Lösning}
\author{Niil Öhlin\footnote{niil.94@hotmail.com}\\
	Luleå tekniska universitet\\
	971 87 Luleå, Sverige
}
\date{\today}
\maketitle
\begin{abstract}
	Uppgiften går att hitta på kursens hemsida, lösningen är en av
	exempellösningarna
\end{abstract}

\newcommand{\ex}[1]{
	e^{-#1 \cdot x^2}
}
\newcommand{\eax} {
	\ex{a}
}
\begin{enumerate}
	\item[14] Lösning
		\begin{enumerate}
			\item 
				\begin{align*}
					&y = 0.5 e^{-x^2} \\
					&\text{lutningen}  = y' \\
					&y' = - x \cdot \ex{} \\
					&y'(0.8) = - 0.42 
				\end{align*}
			\item 
				\begin{align*}
					&y = 0.5 \cdot \eax \\
					&y' = -ax \cdot \eax \\
					&\text{Lutningen är störst då } y' 
							\text{är störst alltså där } y'' = 0 \\
					&y' = ax \cdot \eax \\
					&f(x) = -ax \qquad   f'(x) = -a \\
					&g(x) = \eax  \quad g'(x) = -2ax \cdot \eax \\
					y'' &= -ax \cdot \eax + 2 a^2 e^2 \cdot \eax \\
					&= a \cdot \eax(2ax^2 -1) \\
					&a \cdot \eax > 0 \\
					0 &= 2ax^2 - 1 \\
					x &= \sqrt{\frac 1 {2a}}
				\end{align*}
			\item
				\begin{align*}
					&a = \frac 1 {2x^2} = \frac 1 2 = 0.5 \\
					&\text{Svar: } a = 0.5
				\end{align*}
		\end{enumerate}
\end{enumerate}

\end{document}

