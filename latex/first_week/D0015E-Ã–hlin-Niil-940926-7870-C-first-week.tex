\documentclass{article}
\usepackage[swedish]{babel}
\usepackage[utf8]{inputenc}

\begin{document}
\title{Första Läsveckan}
\author{Niil Öhlin}
\date{}
\maketitle

Grundläggande om datorn har gåtts igenom. Bland annat dess komponenter och hur
de fungerar. Praktisk examinering har också genomförts. Där studenterna
fick plocka isär en dator för att identifiera dess interna komponenter.

Studenterna har även introducerats till datorlabbsalarna och fått uppleva en
unixmiljö. De har också haft en föreläsning av en av professorerna på
universitetet där det gicks lite om honom igenom och lite om internets framtid
inklusive html5.

Det är intressant att se hur första läsveckan speglar svenska skolsystemet så
väl.
Eftersom progammering, eller datakunskap av något slag inte är en obligatorisk
del i svenska skolsystemet är det självklart att universitetet inte kan anta att
studenterna har läst något sådant. Också märks en enorm skillnad på förkunskaper
innom teknikområderna. Till skillnad ifrån matematikunervisningen, som har
extrem standard i form utav nationella prov och linjära kurser. Har
teknikundervisningen olika, oberoende kurser som ingen utav dem är krav för
att komma in på datateknik. Men som gör skillnad för de som börjar på
programmet. Det är lustigt att se att till exempel om en person vill
bli matematiker, fysiker, kemist eller biolog så är förkunskaper
inom området en självklarhet. Men om hen vill bli teknolog, så behövs bara
förkunskaper innom andra ämnen, som fysik och matte.

Emellertid så ser jag fram emot kommande, mer utmanande, större uppgifter
som kräver lite tekniskt kunnande, i förhållande till de relativt ensidiga
uppgifter vi haft hittils.
\end{document}
